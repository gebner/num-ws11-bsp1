\documentclass{scrartcl}
\usepackage[utf8]{inputenc}
\usepackage[ngerman]{babel}
\usepackage[fleqn]{amsmath}
\usepackage{amssymb}
\usepackage{parskip}
\usepackage{graphicx}

\usepackage{listings}
\lstset{language=Octave, basicstyle=\tt, tabsize=8,
  breaklines=true, caption=\texttt\lstname, captionpos=b}
\DeclareFontShape{OT1}{cmtt}{bx}{n}
{<5><6><7><8><9><10><10.95><12><14.4><17.28><20.74><24.88>cmttb10}{}

\begin{document}

\title{Numerische Mathematik UE -- 1. Projekt, Teilprojekt 3}
\author{Gabriel Ebner, Johannes Hafner}
\maketitle

\section{Theoretischer Teil}

Um den Verfahrensfehler bei der numerischen Differentiation durch Interpolation
mit einem Polynom zu bestimmen, wenden wir Satz 4.3 aus der Vorlesung an:

\begin{eqnarray*}
|y'(t) - p'(t)| &\leq& 1 \cdot
    \frac{\|f^{(l+n+1)}\|_\infty}{(l+n)!}\; ((t+nh) - (t-lh))^{l+n} \\
  &=& \|f^{(l+n+1)}\|_\infty\; \frac{(l+n)^{l+n}}{(l+n)!}\; h^{l+n}
\end{eqnarray*}

Somit ist \(|y'(t)-p'(t)| = O(h^6)\), wenn nur \(l+n \geq 6\).  Für \(l=n=3\)
erhalten wir konkret \(|y'(t)-p'(t)| \leq 65\; \|f^{(7)}\|_\infty\; h^6\).

\section{Experimenteller Teil}
\subsection{Test der Funktion}

Wir haben unser Programm an mehreren Funktionen getestet.  Als erstes (siehe
Abbildung \ref{fig:exp}) haben wir ein relativ gut-konditioniertes Problem
gewählt, nämlich die Exponentialfunktion an der Stelle \(2\), hier erwarteten
wir eine schnelle Konvergenz.
Die Funktionen l=n wurde von 1 bis 5 gewählt.
Zunächst fächern sich die Fehlerferläufe auf, da die Konvergenzordnung jeweils \(O(h^{(n+l)})\) beträgt , bis die Rechenfehler die Verfahrensfehler überwiegen. In diesen Fällen haben alle Funktionen ähnliche Fehler.
Ab n=l=3 wird eine Genauigkeit von \(10^{-15}\) erreicht.

\begin{figure}[!htb]
\centering
\includegraphics{fig_exp_internal.pdf}
\caption{\(e^x\) differenziert an der Stelle \(x=2\)}
\label{fig:exp}
\end{figure}


Im Falle einer Polynomiellen Funktion (Abbildungen \ref{fig:lin} und \ref{fig:poly6}) wird sofort eine Genauigkeit von \(10^{-15}\) erreicht, wenn die Funktion durch ein Polynom von ausreichendem Grad aproximiert wird.
Dadurch kann im Falle der linearen Funktion der Rechenfehler gut beobachtet werden.


\begin{figure}[!htb]
\centering
\includegraphics{fig_lin_internal.pdf}
\caption{\(\frac x {10} + 5\) differenziert an der Stelle \(x=1\)}
\label{fig:lin}
\end{figure}

\begin{figure}[!htb]
\centering
\includegraphics{fig_poly6_internal.pdf}
\caption{\(x^6+x^5+x^4+x^3+x^2+x^1+x+1\) differenziert an der Stelle \(x=2\)}
\label{fig:poly6}
\end{figure}


Zu guter letzt(Abbildung \ref{fig:scaledsin})
haben wir eine Funktion gesucht, bei der die Konvergenz erst mit sehr kleinen
Schrittweiten einsetzt, entschieden haben wir uns für einen skalierten Sinus,
nämlich \(x \mapsto 10^{-5}\, \sin(10^5\, x)\) an der Stelle \(1\).
Erst wenn die Schreittweite kleiner als die Periode ist, setzt eine sinnvolle Konvergenz ein. In diesem Fall wird auch nur eine Genauigkeit von \(10^{-6}\) erreicht.


\begin{figure}[!htb]
\centering
\includegraphics{fig_scaledsin_internal.pdf}
\caption{\(10^{-5}\, \sin(10^5\, x)\) differenziert an der Stelle \(x=1\)}
\label{fig:scaledsin}
\end{figure}


\subsection{Vergleich mit den anderen Teilprojekten}

Anhand der Exponentialfunktion (Abbildung \ref{fig:exp_ns}) zeigt sich der Unterschied in den Koeffizienten bei der Konvergenz.

\begin{figure}[!htb]
\centering
\includegraphics{fig_exp_noscale.pdf}
\caption{\(e^x\) ohne Skalierung}
\label{fig:exp_ns}
\end{figure}

Da bei den verschiedenen Konvergenzverfahren \(h\), \(2h\), \(3h\) bzw. \(h\), \(\frac h 2\), \(\frac h 4\),\dots
verwendet wird, haben wir uns entschieden, zur besseren Vergleichbarkeit eine Skalierung der Schrittweiten durchzuführen.
Dabei wurde so umskaliert, das jeweils die Entfernung zwischen dem Punkt, an dem differenziert wird, und dem nächstgelegenen Punkt h beträgt.
Dadurch entsteht die Abbildung \ref{fig:exp_s}.

\begin{figure}[!htb]
\centering
\includegraphics{fig_exp.pdf}
\caption{\(e^x\) mit Skalierungs}
\label{fig:exp_s}
\end{figure}

\begin{figure}[!htb]
\centering
\includegraphics{fig_scaledsin.pdf}
\caption{\(10^{-5}\, \sin(10^5\, x)\) mit Skalierungs}
\label{fig:sin_s}
\end{figure}

Sowohl bei der Exponentialfunktion als auch beim Sinus zeigt sich, dass die Verfahren im Ergebnis zu sehr ähnlichen Fehlerverläufen und Genauigkeiten gelangen.


\subsection{Betriebsmittel}

Die Berechnungen wurden mit double precision unter Octave 3.2.4 und Ubuntu
Oneiric (amd64) auf einem AMD Phenom II X4 965 Prozessor durchgeführt.

\subsubsection{Quelltext}

\lstinputlisting{nd.m}

\end{document}
