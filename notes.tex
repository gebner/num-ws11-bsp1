\documentclass{scrartcl}
\usepackage[utf8]{inputenc}
\usepackage[ngerman]{babel}
\usepackage[fleqn]{amsmath}
\usepackage{amssymb}
\usepackage{parskip}
\usepackage{graphicx}

\usepackage{listings}
\lstset{language=Octave, basicstyle=\tt, tabsize=8,
  breaklines=true, caption=\lstname, captionpos=b}
\DeclareFontShape{OT1}{cmtt}{bx}{n}
{<5><6><7><8><9><10><10.95><12><14.4><17.28><20.74><24.88>cmttb10}{}

\begin{document}

\title{Numerische Mathematik UE -- 1. Projekt, Teilprojekt 3}
\author{Gabriel Ebner, Johannes Hafner}
\maketitle

\section{Theoretischer Teil}

Um den Fehler bei der numerischen Differentiation durch Interpolation mit einem
Polynom zu bestimmen, wenden wir Satz 4.3 aus der Vorlesung an:

\begin{eqnarray*}
|y'(t) - p'(t)| &\leq& 1 \cdot
    \frac{\|f^{(l+n+2)}\|_\infty}{(l+n+1)!}\; ((t+nh) - (t-lh))^{l+n+1} \\
  &\leq& \|f^{(l+n+2)}\|_\infty\; \frac{(l+n)^{l+n+1}}{(l+n+1)!}\; h^{l+n+1}
\end{eqnarray*}

Somit ist \(|y'(t)-p'(t)| = O(h^6)\), wenn nur \(l+n \geq 5\).  Für \(l=2\) und
\(n=3\) erhalten wir konkret \(|y'(t)-p'(t)| \leq 22\; \|f^{(7)}\|_\infty\; h^6\).

\section{Experimenteller Teil}

\begin{figure}[!htb]
\centering
\includegraphics{fig_exp.pdf}
\caption{\(e^x\) differenziert an der Stelle \(x=2\)}
\label{fig:exp}
\end{figure}

\begin{figure}[!htb]
\centering
\includegraphics{fig_scaledsin.pdf}
\caption{\(10^{-5}\, \sin(10^5\, x)\) differenziert an der Stelle \(x=1\)}
\label{fig:scaledsin}
\end{figure}

\subsection{Betriebsmittel}

\subsubsection{Quelltext}

\lstinputlisting{nd.m}

\end{document}
